\chapter{Índice de Estabilidad}

En este capítulo se introducirá la noción de índice de estabilidad de un grafo con peso, denotado $S(G,w)$.

Para esto en primer lugar se introduce una generalización de la propiedad de bipartito a grafos con pesos positivos y negativos
en las aristas.

\begin{defn}[Grafo balanceado]
    
\end{defn}

Se define el índice de estabilidad de un grafo $(G,w)$ como el número
\begin{equation*}
    S(G,w) = -n - \overline{d} + 2\overline{m^+} - 2\overline{m^-} -4\cdot \rho(G,w)
\end{equation*}

Donde $n$ es el número de nodos, $\overline{d}$ es la suma de los pesos de los loops, $\overline{m^+}$ es la suma de los pesos
de las aristas positivas, $\overline{m^-}$ es la suma de los pesos de las aristas negativas y $\rho(G,w)$ es el índice de 
antifrustración del grafo.